\documentclass[]{book}
\usepackage{lmodern}
\usepackage{amssymb,amsmath}
\usepackage{ifxetex,ifluatex}
\usepackage{fixltx2e} % provides \textsubscript
\ifnum 0\ifxetex 1\fi\ifluatex 1\fi=0 % if pdftex
  \usepackage[T1]{fontenc}
  \usepackage[utf8]{inputenc}
\else % if luatex or xelatex
  \ifxetex
    \usepackage{mathspec}
  \else
    \usepackage{fontspec}
  \fi
  \defaultfontfeatures{Ligatures=TeX,Scale=MatchLowercase}
\fi
% use upquote if available, for straight quotes in verbatim environments
\IfFileExists{upquote.sty}{\usepackage{upquote}}{}
% use microtype if available
\IfFileExists{microtype.sty}{%
\usepackage{microtype}
\UseMicrotypeSet[protrusion]{basicmath} % disable protrusion for tt fonts
}{}
\usepackage[margin=1in]{geometry}
\usepackage{hyperref}
\hypersetup{unicode=true,
            pdftitle={FlipAround},
            pdfauthor={MDSI student community},
            pdfborder={0 0 0},
            breaklinks=true}
\urlstyle{same}  % don't use monospace font for urls
\usepackage{natbib}
\bibliographystyle{apalike}
\usepackage{longtable,booktabs}
\usepackage{graphicx,grffile}
\makeatletter
\def\maxwidth{\ifdim\Gin@nat@width>\linewidth\linewidth\else\Gin@nat@width\fi}
\def\maxheight{\ifdim\Gin@nat@height>\textheight\textheight\else\Gin@nat@height\fi}
\makeatother
% Scale images if necessary, so that they will not overflow the page
% margins by default, and it is still possible to overwrite the defaults
% using explicit options in \includegraphics[width, height, ...]{}
\setkeys{Gin}{width=\maxwidth,height=\maxheight,keepaspectratio}
\IfFileExists{parskip.sty}{%
\usepackage{parskip}
}{% else
\setlength{\parindent}{0pt}
\setlength{\parskip}{6pt plus 2pt minus 1pt}
}
\setlength{\emergencystretch}{3em}  % prevent overfull lines
\providecommand{\tightlist}{%
  \setlength{\itemsep}{0pt}\setlength{\parskip}{0pt}}
\setcounter{secnumdepth}{5}
% Redefines (sub)paragraphs to behave more like sections
\ifx\paragraph\undefined\else
\let\oldparagraph\paragraph
\renewcommand{\paragraph}[1]{\oldparagraph{#1}\mbox{}}
\fi
\ifx\subparagraph\undefined\else
\let\oldsubparagraph\subparagraph
\renewcommand{\subparagraph}[1]{\oldsubparagraph{#1}\mbox{}}
\fi

%%% Use protect on footnotes to avoid problems with footnotes in titles
\let\rmarkdownfootnote\footnote%
\def\footnote{\protect\rmarkdownfootnote}

%%% Change title format to be more compact
\usepackage{titling}

% Create subtitle command for use in maketitle
\newcommand{\subtitle}[1]{
  \posttitle{
    \begin{center}\large#1\end{center}
    }
}

\setlength{\droptitle}{-2em}
  \title{FlipAround}
  \pretitle{\vspace{\droptitle}\centering\huge}
  \posttitle{\par}
  \author{MDSI student community}
  \preauthor{\centering\large\emph}
  \postauthor{\par}
  \date{}
  \predate{}\postdate{}

\usepackage{booktabs}
\usepackage{amsthm}
\makeatletter
\def\thm@space@setup{%
  \thm@preskip=8pt plus 2pt minus 4pt
  \thm@postskip=\thm@preskip
}
\makeatother

\usepackage{amsthm}
\newtheorem{theorem}{Theorem}[chapter]
\newtheorem{lemma}{Lemma}[chapter]
\theoremstyle{definition}
\newtheorem{definition}{Definition}[chapter]
\newtheorem{corollary}{Corollary}[chapter]
\newtheorem{proposition}{Proposition}[chapter]
\theoremstyle{definition}
\newtheorem{example}{Example}[chapter]
\theoremstyle{remark}
\newtheorem*{remark}{Remark}
\begin{document}
\maketitle

{
\setcounter{tocdepth}{1}
\tableofcontents
}
\chapter{Welcome}\label{welcome}

MDSI is unique in its approach and feel. MDSI is a `boutique degree'
which means we are a small tight-knit data family which means the
contacts you walk out (really) knowing are going to be more valuable
than the skills you learn. In terms of content, our point of difference
is the innovation in our name. We take our innovation component as
seriously as data science, and is ingrained in everything that's taught.
For us, a data science degree was our innovation (we were the first of
its kind in Australia), and in the rapidly changing context that is
data, the ability to innovate and adapt is a pretty great point of
difference for you too.

Data science is a collaborative discipline. Students in the MDSI program
get hands on experience of working in teams to formulate and solve
real-life data science problems. Most courses focus on techniques to
solve problems, but spend very little time (if any) on how problems
should be formulated. The MDSI program is structured in a way that helps
students learn this tacit, but crucial skill.

Another important aspect of data science is that it is a rapidly
evolving field. A data scientist must therefore be able to stay current
with developments in the field. The MDSI program, with its emphasis on
critical self-learning, prepares students to be lifelong learners.

Welcome, and good luck on your MDSI journey

\section{Checklist of things to do}\label{checklist-of-things-to-do}

Getting started on your MDSI journey can be somewhat overwhelming. So to
help you ease into life as an MDSI student, the following checklist will
help you to get up and running as painlessly as possible.

\begin{itemize}
\tightlist
\item
  Do your statistics pre-flight test
\item
  Activate UTS student email
\item
  Forwarded UTS student email if required
\item
  Review Subject Outlines
\item
  Activate and personalise CICAround
\item
  Do your CLARA profile
\item
  Log into Diigo
\item
  Join the Slack Channel
\item
  Log into Review
\item
  Log into SPARK
\item
  Test your Google Drive
\item
  Test your Office 365 Drive
\item
  Download R \& R Studio
\item
  Download Python \& Rodeo
\item
  Download Tableau
\item
  Download KNIME
\end{itemize}

\section{Pre-flight tests}\label{pre-flight-tests}

MDSI statistics pre-flight test:
\url{http://www.uts.edu.au/future-students/analytics-and-data-science/essential-information/mdsi-statistics-pre-flight-test}

\chapter{The data science mindset}\label{the-data-science-mindset}

\section{CLARA}\label{clara}

Each person has their own learning preferences and habits of mind that
shape their response to challenges and learning opportunities.

CLARA (Crick LeArning for Resilient Agency) is a tool used to prompt
reflection on a multidimensional construct called ``Learning Power''
with eight dimensions: curiosity, creativity, sense making, belonging,
collaboration, hope and optimism, mindful agency and openness to change.
The UTS Graduate Attributes have a strong resonance with these
dimensions. The CLARA tool is used as part of MDSI activities, aiming to
help the students maximise their development results through
understanding themselves better, namely with regards to their approach
to learning and challenges.

The tool is survey-based and provides a profile-style feedback covering
each of the following dimensions: Curiosity - Wanting to get beneath the
surface \& find out more - Always wondering why and how Creativity -
Using my intuition \& imagination to generate new ideas \& knowledge -
Taking risks \& playing with ideas and artefacts to arrive at new
solutions Sense making - Making connections between what I already know
\& new information \& experience - Making meaning by linking my story,
my new learning \& my purpose Belonging - Being part of a learning
community at work, at home, in education \& in my social networks -
Knowing I have social resources to draw on when I need them
Collaboration - Being able to work with others, to collaborate and
co-generate new ideas and artefacts - Being able to listen and
contribute productively to a team Hope and optimism - Having the
optimism \& hope that I can learn \& achieve over time - Having a growth
mindset; believing I can generate my own new knowledge for what I need
to achieve Mindful agency - Taking responsibility for my own learning
over time through defining my purposes, understanding and managing my
feelings, - Knowing how I go about learning \& planning my learning
journey carefully Openness to change - An emotional orientation of being
open \& ready to invest in learning, having flexible self-belief,
willing to persist \& manage any self-doubt - A necessary prerequisite
for developing resilience in learning

Here is an example of an output from CLARA, showing the resulting
profile, based on the responses provided in the survey.

\emph{Insert images here}

Learning profiles can change over time, so MDSI conducts two sets of
CLARA tests, to allow the students to compare their profile changes and
reflect on their development progress.

CLARA survey will be organised by MDSI and each student will receive a
link and instructions on how to utilise this tool.

\section{Data Science Professional
Competencies}\label{data-science-professional-competencies}

Competency can be defined as ``effective application of skill, knowledge
and abilities to on-the-job-behaviour and capability to perform to job
requirements''. MDSI supports the ongoing development professional
development efforts and offers a tool that can help the students
evaluate their skills and abilities in various domains associated with
the data science professional competencies. Students are encouraged to
utilise the tool to identify the key competencies associated with their
individual career aspirations. For each relevant competency, students
evaluate their current competency levels, identify any gaps and use the
information to create a set of actions that would form their
professional development plan. The competencies model is introduced to
students as part of 36100 (Data Science and Innovation) subject through
a self-assessment exercise. The competencies are divided into two
groups: technical and managerial, describing the following competency
(proficiency) levels for each domain: Beginner, Competent, Advanced and
Expert.

Technical: - Mathematics and statistics - Programing and scripting
languages - Databases and data storage - Computing systems, platforms,
security, integration - Data mining - Data modelling - Analytics,
predictive modelling and machine learning - Data visualisation -
Business analysis and interpretation - Product development

Interpersonal/managerial: - Creativity - Communication - Data strategy -
Line management - Data management and governance - Facilitation and
presentation - Project management

Competency levels: \textbf{Beginner:} able to assist and perform simple
tasks \textbf{Competent:} able to perform tasks independently
\textbf{Advanced:} able to perform complex tasks \textbf{Expert:} able
to perform complex transformative, strategic or trans-disciplinary tasks

The competencies assessment exercise aims to prompt the students to take
a proactive attitude to their professional development efforts and
effectively apply their analytical skills, dedication and
professionalism in managing their career objectives.

The competencies assessment exercise covers the following steps: 1.
Evaluate your current competency level for each domain on the list 2.
Choose a set of domains (no more than 6) that are relevant to your
planned development for this subject, your course and your career goals.
3. Identify the goal competency levels for the selected domains and
describe related professional development outcomes that support your
assessment 4. Analyse your development outcomes in the context of your
career goals and identify the gaps between your current and goal
competency levels 5. Develop a set of actions needed to achieve desired
level of competencies and bridge the identified gap 6. Provide feedback
and suggestions for the improvement of the current list of competencies,
descriptions etc.

\section{Ethics amd Privacy}\label{ethics-amd-privacy}

Its important to understand that security, privacy and ethics are three
different things, although heavily intertwined in the `internet of
things'.

What is ethical when it comes to data and the internet of things? Is
privacy having a login or not being identifiable as an individual?

The world of Ethics and Privacy is changing, similar to the definition
that now includes much more than it did a decade ago. Computer security
like a login is no longer sufficient to providing protection of privacy
which is more focused on ensuring that only people who should have the
authority to access your information should be able to.

Current Privacy legislation addresses control and authentication
processes of whom can access your information via direct disclosures and
how this information should be stored by the party who is collecting
this information, it does not address disclosures that can be made based
on inferences that can be drawn from big data of which your information
is a part. Is the value or conclusions that could be drawn from
information as part of big data considered private information?

A sensible framework in relation to Ethics and Privacy where data is
concerned is highlighted in the Belmont report which identifies two
rules to consider ``(1) do not harm and (2) maximize possible benefits
and minimize possible harms.''

A big ethical dilema of late is the rich data sources that various
provider hold, that if pooled together will strip all possibility of
anonymity.

For more on this read:

\url{http://www.tandfonline.com/doi/full/10.1080/08900523.2014.863126?src=recsys}
\url{http://libres.uncg.edu/ir/uncg/f/N_Kshetri_Big_2014.pdf}

\section{Digital Footprint}\label{digital-footprint}

Your digital footprint is the name given to the data that is recorded
about you all day every day. It can be the time and phone number of
someone that you called, the mobile phone tower that you were connected
to at the time of making the call and how long you spoke for. It is the
IP address of your computer when you connect to the internet. It is the
list of items you pay for when you go through the checkout at the
supermarket and the eftpos card number you used to pay for the items. It
is the surveillance footage you appear in when you move through
monitored public spaces. It is stories you `like' or share on social
media sites. It is the journeys that your GPS navigation stores about
your travels. It is every email you send and every click you make when
you browse the internet.

Your digital footprint is the inescapable record of your existence by
doing nothing more than living your life. It is an important aspect of
modern society as many services that you enjoy depend on the data you
generate in order to provide critical services. A bank can't tell you
how much money you have without keeping record of your bank
transactions. For good or for evil, this data comes embedded with far
more information about you. By looking at the kinds of things you spend
your money on or the businesses that you spend your money at and the
time of day that you spend your money there, it can be determined where
you live and where you work.

As an MDSI student, you will learn to think critically and ethically
about data collection and how it can be used for good and for evil. The
best place to start your thinking is with your own digital footprint,
become aware of how big it is and how you feel about it.

It's important to note that very little permission is sought on data
collection and when it is sought, very little education is provided in
terms of the use of that data. Very few providers who collect data
clarify or specify what the data they collect is used for.

You are responsible for your digital footprint. Generate it wisely.

\section{Opportunity for overseas
exchange}\label{opportunity-for-overseas-exchange}

Some great opportunities exist within MDSI with our Program Director
having many contacts in many other countries which enable us to be able
to explore greater opportunities for overseas exchange.

You need to do a few things before this opportunity is explored as set
out by the Program Director to ensure for an easier way forward if this
is an opportunity you want to explore.

\section{Electives}\label{electives}

You need to select four electives during your MDSI course. These
electives should be selected to assist you in your growth as a student
and as a data science professional. These subjects enable you to add to
your toolbox of where you are heading with your journey.

Electives can be selected from any school however you will still be
subjected to the pre-requisites for any possible subject, so it will
depend on the requirements of the subject.

We suggest that when you apply for a subject with a prerequisite that
you also apply for a waive of the prerequisite if the prerequisite is a
subject you are familiar with but have not done with UTS and get
exemption for that requisite.

This is not always easy, or approved and is subject to each School's
internal views or policies. It is definitely a consideration to take.

You can apply for the subject ( and a waiver of prerequisites if
required) early as CIC is not limited by inter-faculty time
restrictions.

Our best tip is : get in early.

\chapter{\texorpdfstring{A `survival guide' to
MDSI}{A survival guide to MDSI}}\label{a-survival-guide-to-mdsi}

\section{First steps}\label{first-steps}

\subsection{Your UTS email:}\label{your-uts-email}

First and foremost you need to activate your UTS email address. All
official communications from UTS, subject notifications, MDSI
newsletters etc will be sent to this email address. You need to activate
your email address before you can access other UTS systems.

\textbf{Activate your UTS Student email:}

\begin{itemize}
\tightlist
\item
  Navigate to
  \url{https://email.itd.uts.edu.au/webapps/myaccount/activation/} and
  follow the steps to activate your UTS student email account.
\item
  \textbf{Protip:} If you don't want to login frequently to check if you
  have mail, simply setup a email forwarding to an email address of your
  choice via the settings page after logging in.
\end{itemize}

\subsection{Get your subject outlines:}\label{get-your-subject-outlines}

MDSI uses a variety of systems for online teaching and learning.
UTSOnline and CICAround are the two primary environments for you to
familiarise yourself with.

The first thing you need to do after activating your email address is to
login to UTSOnline, access your subjects and find your subject outline.
Your subject outline contains everything you need to know about your
subject for the coming semester. It includes the contact information for
your subject co-ordinator, important dates, assessment descriptions and
much more. In most cases you can find the answer to any question you
might have about your subject addressed in the subject outline.

\textbf{Find your subject outlines in UTSOnline:}

\begin{itemize}
\tightlist
\item
  Login to UTSOnline at \url{https://online.uts.edu.au/} using your
  student ID number and the password you setup for your UTS email
  account.
\item
  Access your subjects by clicking on your subject name
\item
  Download your subject outline by clicking on the link titled `Subject
  Outline' in the left side menu, then click on the subject outline link
  on the page.
\end{itemize}

\section{Technology}\label{technology}

\subsection{Join the MDSI community:}\label{join-the-mdsi-community}

Your next stop should be CICAround. Here you will connect with your
peers in an academic capacity. There are discussion forums for your
subjects where you can post questions. CICAround most notably is where
you will go to blog about your experiences throughout your MDSI journey.
The first step is to activate your blog. Then you can browse through the
blogs of your new MDSI family and read about their experiences and the
things they have learnt.

\textbf{Activate and personalise your CICAround profile:} - Navigate to
\url{https://ca.uts.edu.au/using-ca/} - Watch the welcome video then
login to CICAround using your student ID and password. - Put up your
first CICAround Blog post

\subsection{Join the MDSI chatter:}\label{join-the-mdsi-chatter}

Slack has proven to be a very useful tool so far. It is completely
driven by the student community and is where the MDSI student community
goes to socialise, organise BBQs, ask each other for technical help and
share useful resources. If you need a quick answer, Slack is the place
to go.

\textbf{Join the Slack Channel} - You can download the Slack application
from \url{https://slack.com/downloads} - You can also get the app for
IOS, Android and Windows phones. - You do not need to pay for a
subscription. - Signup to the MDSI group at:
\url{https://utsmdsi.slack.com/}

\subsection{Diigo}\label{diigo}

A collection of resources contributed by the MDSI community.

\textbf{Join the Diigo group} - simply create a Diigo account and
request access. \url{https://groups.diigo.com/group/cic_mdsi}

Frequently used search tags include: \textgreater{}
\emph{``DSI,DAM,Data,big\_data,case,studies,visualization,teaching\_tools,statistics,stats-thnkg,privacy,Algorithms,ethicsVSD,realworldDM,video,Analytics,human-machine,history,data,mining,Data\_science,cisco,r,IoE,TEDtalks,values,QSProject,Algorithmic,Accountability,industry,sociotechnical,systems,AI,podcast,professional\_practice,portfolio,storytelling,bbc,QS,DMonline,innovation,humanismprofessional,development,open\_data,speculative\_futures,RealWorld,ted,transdisciplinarity,creativity,algorithm,sociotechnical,BowkerStar,futuresgender,challenge,data-sets,accountability,digital\_futures,tools,DM,reading,DVN,equality,infographic''}
\textgreater{}

\subsection{Review}\label{review}

Review is an assessment tool that is used to mark your work, give you
feedback about your work and for you to develop a sense of what is
expected by marking your own work before it is assessed by teaching
staff.

\subsection{SPARK}\label{spark}

SPARK is an acronym for Self and Peer Assessment Review Kit. This tool
has been developed to assist with in class activities as well as being
able to self and peer assess assignments. Its also one of the few tools
that make group assessments/work easier, in particular to marking.

\subsection{Google / Office 365}\label{google-office-365}

Your university account allows you access to Google Apps and Office 365.

Google Apps access does not include Gmail. You can not login to your UTS
google apps account via gmail or if you are already logged in with a
personal gmail account, you will need to log out completely from gmail.
Once you have done this, you can log in using your student email
address. This will revert you to a UTS login page. Use your UTS student
number and password and it will revert you back to the Google Drive, but
you will be logged into the drive.

Similarly you can mimic the same steps for Office 365.

\subsection{R / R Studio}\label{r-r-studio}

`R' is a coding language used by most of the data science community.
RStudio is a software program or `Integrated Development Environment'
(IDE) that makes working with the R language ALOT easier. The
programming environment is really flexible as it allows you the joy of
working in a notebook format, scripting, markdown and publishing your
work as a PDF.

You will use R in many of your subjects and being able to use it well
will give you a serious edge over your classmates and competitors at
hackathons.

\textbf{Download and install R \& RStudio free} - Download and install
the R language: \url{https://cran.rstudio.com/} - Download and install
RStudio IDE: \url{https://www.rstudio.com/products/rstudio/download/}

\textbf{Libraries well worth their weight in gold:} - tidyverse -
\url{http://tidyverse.org/} A collection of libraries that make data
analysis easier - readr - \url{http://readr.tidyverse.org/} for reading
all kinds of data formats - stringr -
\url{http://stringr.tidyverse.org/} for working with text - ggplot2 -
\url{http://ggplot2.tidyverse.org/} for visualising data - tidyr -
\url{http://tidyr.tidyverse.org/} for creating tidy data - dplyr -
\url{https://github.com/hadley/dplyr} for manipulating data - caret -
\url{http://topepo.github.io/caret/index.html} for creating predictive
models - Bookdown - \url{https://bookdown.org/} for creating beautiful
documents

There are many resources to get you started in doing data science with
R. Refer to the resources section for more information.

\subsection{Python / Rodeo / Jupyter
Notebook}\label{python-rodeo-jupyter-notebook}

Python is a general purpose coding language widely used by the data
science community. A great place to start is with the Anaconda from
Continuum Analytics : \url{https://www.continuum.io/downloads}

Python for data science is commonly used in a notebook format. To this
end Jupyter notebooks will become a familiar friend. Fortunately it is
included as part of Anaconda. For more info, refer to the resources
section.

If you prefer an `R' style IDE, then Rodeo by Yhat is for you.
\url{https://www.yhat.com/products/rodeo}

If you prefer a traditional IDE, you can get a free license for PyCharm
(as well as all their other products) from JetBrains using your student
details: \url{https://www.jetbrains.com/student/}

\textbf{Libraries well worth their weight in gold:} - Numpy -
\url{http://www.numpy.org/} for working with numerical arrays - Scipy -
\url{https://www.scipy.org/scipylib/index.html} for scientific computing
with python - Matplotlib - \url{http://matplotlib.org/} for visualising
data - Seaborn - \url{http://seaborn.pydata.org/index.html} for
statistical visualisation - Pandas - \url{http://pandas.pydata.org/} for
working with data - Statsmodels -
\url{http://www.statsmodels.org/stable/index.html} for creating
statistical models - Scikit-Learn -
\url{http://scikit-learn.org/stable/} for doing machine learning with
python - Tensorflow - \url{https://www.tensorflow.org/} `deep learning'
with python

Note: Python comes in two different flavours: 2.7 and 3.x. You can start
with either version, but it is worth learning what the subtle
differences are (eventually).

\textbf{A couple of blog posts to help you choose between R and Python:}
- R vs Python for Data Science: The Winner is
\url{http://www.kdnuggets.com/2015/05/r-vs-python-data-science.html} - R
vs Python for Data Science: Summary of Modern Advances
\url{https://elitedatascience.com/r-vs-python-for-data-science}

\subsection{Tableau}\label{tableau}

Tableau is a tool for visualising data. It is quite powerful in its
ability to connect to a variety of data sources both on your computer
and through the internet. It is also relatively intuitive to use.

As a student you can apply to the company for a free license to their
commercial desktop version.
\url{https://www.tableau.com/academic/students}

\subsection{KNIME}\label{knime}

KNIME Analytics Platform is an open source solution that enables quick,
fast data driven designs for machine learning. Its a visual tool to
learn and use when you need to get the job done quickly (without writing
any code) and need to create algorithms quickly but you don't have the
time to learn the mathematics behind the algorithms. Its friendly and
easy to use to find the hidden `story' in the data.

Go to \url{https://www.knime.org/knime-analytics-platform} to download
KNIME for free.

\subsection{Rapidminer}\label{rapidminer}

Rapidminer is another visual tool for doing data analysis, modelling and
machine learning. You can get access to their commercial tools using
your student status from
\url{https://rapidminer.com/educational-program/}

\subsection{QGIS}\label{qgis}

QGIS is a really nice open source tool for working with geospatial data.
To get started just head over to
\url{http://www.qgis.org/en/site/index.html}

\subsection{Github}\label{github}

Github is a fantastic tool to get familiar with. It is a great place to
store code, collaborate with other and even host your own website or
blog.

Github has a really generous collection of free stuff for students. To
claim yours head over to: \url{https://education.github.com/pack}

\section{Writing}\label{writing}

\subsection{Blogs}\label{blogs}

More and more academics and workplaces use blog posts to reach clients,
audiences and share knowledge. Blogs can be useful for many reasons and
is used as a reflective tool for students as well as providing an
opportunity to share any learning.

You can use some tools to turn topics into amazing titles by using
keyword suggesters (\url{http://keywordtool.io}), title generators (
\url{https://www.portent.com/tools/title-maker}), and you can also test
your headlines with the following tool
(\url{http://coschedule.com/headline-analyzer\#})

Tips for new bloggers - Use an eye catching title - In-text links - Use
pictures, pictures speak a thousand words - Keep post to 1000-1500 words
- Use social sharing buttons - Use paragraphs - one idea per paragraph -
Revise and Rewrite - Omit needless words - Use the KISS (Keep It Simple,
Stupid) Principle - Use definite, specific concrete language - direct
and to the point - Write in a way that comes naturally - use your active
voice - Be clear - make it simple to read and understand - Avoid fancy
words - Do not take shortcuts at the cost of clarity

Tips on writing blog posts :
\url{https://problogger.com/how-to-write-great-blog-content/} and
\url{http://www.socialmediaexaminer.com/26-tips-for-writing-great-blog-posts/}

\subsection{White papers}\label{white-papers}

Where do you start with a white paper and what are they? White papers
are originally documents written for government policies however this is
most recently being used by companies and universities to get new
policies and research into the public space.

There are some things to consider when writing a white paper: - Pick a
topic people will want to read or a problem you want to solve - Pick a
generic title that describes the problem at hand - Engage, inform and
convince your reader - Be descriptive and professional - Consider the
audience you are `speaking' to and accommodate for their level of
expertise - Set up a great intro - Emphasize the value you want to or
will create - Decide on a length for the white paper (1-5 pages are the
norm) - Describe the solution you are proposing - Remember a summary
that reviews the problem, solution and result of the outcome - Proofread
your document, and ensure someone else reads it before you
submit/publish it. - Follow the 3-30-3 rule ( you have three seconds to
captures your audience's attention from a glance at your piece, if you
succeed at capturing their attention then you have 30 more seconds to
ensure they continue reading, if you pass the 3-30 rules then your
reader will give you three more minutes to make your point).

\subsection{Assignments}\label{assignments}

You will get an assignment brief for each assignment you have due. Its
is recommended that you review these briefs and that you follow the
detailed instructions set out for you.

\bibliography{packages.bib,book.bib}


\end{document}
