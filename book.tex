\documentclass[]{book}
\usepackage{lmodern}
\usepackage{amssymb,amsmath}
\usepackage{ifxetex,ifluatex}
\usepackage{fixltx2e} % provides \textsubscript
\ifnum 0\ifxetex 1\fi\ifluatex 1\fi=0 % if pdftex
  \usepackage[T1]{fontenc}
  \usepackage[utf8]{inputenc}
\else % if luatex or xelatex
  \ifxetex
    \usepackage{mathspec}
  \else
    \usepackage{fontspec}
  \fi
  \defaultfontfeatures{Ligatures=TeX,Scale=MatchLowercase}
\fi
% use upquote if available, for straight quotes in verbatim environments
\IfFileExists{upquote.sty}{\usepackage{upquote}}{}
% use microtype if available
\IfFileExists{microtype.sty}{%
\usepackage{microtype}
\UseMicrotypeSet[protrusion]{basicmath} % disable protrusion for tt fonts
}{}
\usepackage[margin=1in]{geometry}
\usepackage{hyperref}
\hypersetup{unicode=true,
            pdftitle={FlipAround},
            pdfauthor={MDSI student community},
            pdfborder={0 0 0},
            breaklinks=true}
\urlstyle{same}  % don't use monospace font for urls
\usepackage{natbib}
\bibliographystyle{apalike}
\usepackage{longtable,booktabs}
\usepackage{graphicx,grffile}
\makeatletter
\def\maxwidth{\ifdim\Gin@nat@width>\linewidth\linewidth\else\Gin@nat@width\fi}
\def\maxheight{\ifdim\Gin@nat@height>\textheight\textheight\else\Gin@nat@height\fi}
\makeatother
% Scale images if necessary, so that they will not overflow the page
% margins by default, and it is still possible to overwrite the defaults
% using explicit options in \includegraphics[width, height, ...]{}
\setkeys{Gin}{width=\maxwidth,height=\maxheight,keepaspectratio}
\IfFileExists{parskip.sty}{%
\usepackage{parskip}
}{% else
\setlength{\parindent}{0pt}
\setlength{\parskip}{6pt plus 2pt minus 1pt}
}
\setlength{\emergencystretch}{3em}  % prevent overfull lines
\providecommand{\tightlist}{%
  \setlength{\itemsep}{0pt}\setlength{\parskip}{0pt}}
\setcounter{secnumdepth}{5}
% Redefines (sub)paragraphs to behave more like sections
\ifx\paragraph\undefined\else
\let\oldparagraph\paragraph
\renewcommand{\paragraph}[1]{\oldparagraph{#1}\mbox{}}
\fi
\ifx\subparagraph\undefined\else
\let\oldsubparagraph\subparagraph
\renewcommand{\subparagraph}[1]{\oldsubparagraph{#1}\mbox{}}
\fi

%%% Use protect on footnotes to avoid problems with footnotes in titles
\let\rmarkdownfootnote\footnote%
\def\footnote{\protect\rmarkdownfootnote}

%%% Change title format to be more compact
\usepackage{titling}

% Create subtitle command for use in maketitle
\newcommand{\subtitle}[1]{
  \posttitle{
    \begin{center}\large#1\end{center}
    }
}

\setlength{\droptitle}{-2em}
  \title{FlipAround}
  \pretitle{\vspace{\droptitle}\centering\huge}
  \posttitle{\par}
  \author{MDSI student community}
  \preauthor{\centering\large\emph}
  \postauthor{\par}
  \date{}
  \predate{}\postdate{}

\usepackage{booktabs}
\usepackage{amsthm}
\makeatletter
\def\thm@space@setup{%
  \thm@preskip=8pt plus 2pt minus 4pt
  \thm@postskip=\thm@preskip
}
\makeatother

\usepackage{amsthm}
\newtheorem{theorem}{Theorem}[chapter]
\newtheorem{lemma}{Lemma}[chapter]
\theoremstyle{definition}
\newtheorem{definition}{Definition}[chapter]
\newtheorem{corollary}{Corollary}[chapter]
\newtheorem{proposition}{Proposition}[chapter]
\theoremstyle{definition}
\newtheorem{example}{Example}[chapter]
\theoremstyle{remark}
\newtheorem*{remark}{Remark}
\begin{document}
\maketitle

{
\setcounter{tocdepth}{1}
\tableofcontents
}
\chapter{Welcome}\label{welcome}

MDSI is unique in its approach and feel. MDSI is a `boutique degree'
which means we are a small tight-knit data family which means the
contacts you walk out (really) knowing are going to be more valuable
than the skills you learn. In terms of content, our point of difference
is the innovation in our name. We take our innovation component as
seriously as data science, and is ingrained in everything that's taught.
For us, a data science degree was our innovation (we were the first of
its kind in Australia), and in the rapidly changing context that is
data, the ability to innovate and adapt is a pretty great point of
difference for you too.

Data science is a collaborative discipline. Students in the MDSI program
get hands on experience of working in teams to formulate and solve
real-life data science problems. Most courses focus on techniques to
solve problems, but spend very little time (if any) on how problems
should be formulated. The MDSI program is structured in a way that helps
students learn this tacit, but crucial skill.

Another important aspect of data science is that it is a rapidly
evolving field. A data scientist must therefore be able to stay current
with developments in the field. The MDSI program, with its emphasis on
critical self-learning, prepares students to be lifelong learners.

Welcome, and good luck on your MDSI journey

\section{Checklist of things to do}\label{checklist-of-things-to-do}

Getting started on your MDSI journey can be somewhat overwhelming. So to
help you ease into life as an MDSI student, the following checklist will
help you to get up and running as painlessly as possible.

\begin{itemize}
\tightlist
\item
  Do your statistics pre-flight test
\item
  Activate UTS student email
\item
  Forwarded UTS student email if required
\item
  Review Subject Outlines
\item
  Activate and personalise CICAround
\item
  Do your CLARA profile
\item
  Log into Diigo
\item
  Join the Slack Channel
\item
  Log into Review
\item
  Log into SPARK
\item
  Test your Google Drive
\item
  Test your Office 365 Drive
\item
  Download R \& R Studio
\item
  Download Python \& Rodeo
\item
  Download Tableau
\item
  Download KNIME
\end{itemize}

\section{Pre-flight tests}\label{pre-flight-tests}

MDSI statistics pre-flight test:
\url{http://www.uts.edu.au/future-students/analytics-and-data-science/essential-information/mdsi-statistics-pre-flight-test}

\chapter{The data science mindset}\label{the-data-science-mindset}

\section{CLARA}\label{clara}

\section{Competencies}\label{competencies}

\section{Ethics amd Privacy}\label{ethics-amd-privacy}

Its important to understand that security, privacy and ethics are three
different things, although heavily intertwined in the `internet of
things'.

What is ethical when it comes to data and the internet of things? Is
privacy having a login or not being identifiable as an individual?

The world of Ethics and Privacy is changing, similar to the definition
that now includes much more than it did a decade ago. Computer security
like a login is no longer sufficient to providing protection of privacy
which is more focused on ensuring that only people who should have the
authority to access your information should be able to.

Current Privacy legislation addresses control and authentication
processes of whom can access your information via direct disclosures and
how this information should be stored by the party who is collecting
this information, it does not address disclosures that can be made based
on inferences that can be drawn from big data of which your information
is a part. Is the value or conclusions that could be drawn from
information as part of big data considered private information?

A sensible framework in relation to Ethics and Privacy where data is
concerned is highlighted in the Belmont report which identifies two
rules to consider ``(1) do not harm and (2) maximize possible benefits
and minimize possible harms.''

A big ethical dilema of late is the rich data sources that various
provider hold, that if pooled together will strip all possibility of
anonymity.

For more on this read:

\url{http://www.tandfonline.com/doi/full/10.1080/08900523.2014.863126?src=recsys}
\url{http://libres.uncg.edu/ir/uncg/f/N_Kshetri_Big_2014.pdf}

\section{Digital Footprint}\label{digital-footprint}

Your digital footprint is the name given to the data that is recorded
about you all day every day. It can be the time and phone number of
someone that you called, the mobile phone tower that you were connected
to at the time of making the call and how long you spoke for. It is the
IP address of your computer when you connect to the internet. It is the
list of items you pay for when you go through the checkout at the
supermarket and the eftpos card number you used to pay for the items. It
is the surveillance footage you appear in when you move through
monitored public spaces. It is stories you `like' or share on social
media sites. It is the journeys that your GPS navigation stores about
your travels. It is every email you send and every click you make when
you browse the internet.

Your digital footprint is the inescapable record of your existence by
doing nothing more than living your life. It is an important aspect of
modern society as many services that you enjoy depend on the data you
generate in order to provide critical services. A bank can't tell you
how much money you have without keeping record of your bank
transactions. For good or for evil, this data comes embedded with far
more information about you. By looking at the kinds of things you spend
your money on or the businesses that you spend your money at and the
time of day that you spend your money there, it can be determined where
you live and where you work.

As an MDSI student, you will learn to think critically and ethically
about data collection and how it can be used for good and for evil. The
best place to start your thinking is with your own digital footprint,
become aware of how big it is and how you feel about it.

It's important to note that very little permission is sought on data
collection and when it is sought, very little education is provided in
terms of the use of that data. Very few providers who collect data
clarify or specify what the data they collect is used for.

You are responsible for your digital footprint. Generate it wisely.

\section{Opportunity for overseas
exchange}\label{opportunity-for-overseas-exchange}

Some great opportunities exist within MDSI with our Program Director
having many contacts in many other countries which enable us to be able
to explore greater opportunities for overseas exchange.

You need to do a few things before this opportunity is explored as set
out by the Program Director to ensure for an easier way forward if this
is an opportunity you want to explore.

\section{Electives}\label{electives}

You need to select four electives during your MDSI course. These
electives should be selected to assist you in your growth as a student
and as a data science professional. These subjects enable you to add to
your toolbox of where you are heading with your journey.

Electives can be selected from any school however you will still be
subjected to the pre-requisites for any possible subject, so it will
depend on the requirements of the subject.

We suggest that when you apply for a subject with a prerequisite that
you also apply for a waive of the prerequisite if the prerequisite is a
subject you are familiar with but have not done with UTS and get
exemption for that requisite.

This is not always easy, or approved and is subject to each School's
internal views or policies. It is definitely a consideration to take.

You can apply for the subject ( and a waiver of prerequisites if
required) early as CIC is not limited by inter-faculty time
restrictions.

Our best tip is : get in early.

\chapter{\texorpdfstring{A `survival guide' to
MDSI}{A survival guide to MDSI}}\label{a-survival-guide-to-mdsi}

\section{First steps}\label{first-steps}

\subsection{Your UTS email:}\label{your-uts-email}

First and foremost you need to activate your UTS email address. All
official communications from UTS, subject notifications, MDSI
newsletters etc will be sent to this email address. You need to activate
your email address before you can access other UTS systems.

\textbf{Activate your UTS Student email:}

\begin{itemize}
\tightlist
\item
  Navigate to
  \url{https://email.itd.uts.edu.au/webapps/myaccount/activation/} and
  follow the steps to activate your UTS student email account.
\item
  **Pro-\url{tip:**} If you don't want to login frequently to check if
  you have mail, simply setup a email forwarding to an email address of
  your choice via the settings page after logging in.
\end{itemize}

\subsection{Get your subject outlines:}\label{get-your-subject-outlines}

MDSI uses a variety of systems for online teaching and learning.
UTSOnline and CICAround are the two primary environments for you to
familiarise yourself with.

The first thing you need to do after activating your email address is to
login to UTSOnline, access your subjects and find your subject outline.
Your subject outline contains everything you need to know about your
subject for the coming semester. It includes the contact information for
your subject co-ordinator, important dates, assessment descriptions and
much more. In most cases you can find the answer to any question you
might have about your subject addressed in the subject outline.

\textbf{Find your subject outlines in UTSOnline:}

\begin{itemize}
\tightlist
\item
  Login to UTSOnline at \url{https://online.uts.edu.au/} using your
  student ID number and the password you setup for your UTS email
  account.
\item
  Access your subjects by clicking on your subject name
\item
  Download your subject outline by clicking on the link titled `Subject
  Outline' in the left side menu, then click on the subject outline link
  on the page.
\end{itemize}

\section{Technology}\label{technology}

\subsection{Join the MDSI community:}\label{join-the-mdsi-community}

Your next stop should be CICAround. Here you will connect with your
peers in an academic capacity. There are discussion forums for your
subjects where you can post questions. CICAround most notably is where
you will go to blog about your experiences throughout your MDSI journey.
The first step is to activate your blog. Then you can browse through the
blogs of your new MDSI family and read about their experiences and the
things they have learnt.

\textbf{Activate and personalise your CICAround profile:} - Navigate to
\url{https://ca.uts.edu.au/using-ca/} - Watch the welcome video then
login to CICAround using your student ID and password. - Put up your
first CICAround Blog post

\subsection{Join the MDSI chatter:}\label{join-the-mdsi-chatter}

Slack has proven to be a very useful tool so far. It is completely
driven by the student community and is where the MDSI student community
goes to socialise, organise BBQs, ask each other for technical help and
share useful resources. If you need a quick answer, Slack is the place
to go.

\textbf{Join the Slack Channel} - You can download the Slack application
from \url{https://slack.com/downloads} - You can also get the app for
IOS, Android and Windows phones. - You do not need to pay for a
subscription. - Signup to the MDSI group at:
\url{https://utsmdsi.slack.com/}

\subsection{Diigo}\label{diigo}

A collection of resources contributed by the MDSI community.

\textbf{Join the Diigo group} - simply create a Diigo account and
request access. \url{https://groups.diigo.com/group/cic_mdsi}

Frequently used search tags include: \textgreater{}
``DSI,DAM,Data,big\_data,case,studies,visualization,teaching\_tools,statistics,stats-thnkg,privacy,Algorithms,ethicsVSD,realworldDM,video,Analytics,human-machine,history,data,mining,Data\_science,cisco,r,IoE,TEDtalks,values,QSProject,Algorithmic,Accountability,industry,sociotechnical,systems,AI,podcast,professional\_practice,portfolio,storytelling,bbc,QS,DMonline,innovation,humanismprofessional,development,open\_data,speculative\_futures,RealWorld,ted,transdisciplinarity,creativity,algorithm,sociotechnical,BowkerStar,futuresgender,challenge,data-sets,accountability,digital\_futures,tools,DM,reading,DVN,equality,infographic''
\textgreater{}

\subsection{Review}\label{review}

Review is an assessment tool that is used to mark your work, give you
feedback about your work and for you to develop a sense of what is
expected by marking your own work before it is assessed by teaching
staff.

\bibliography{packages.bib,book.bib}


\end{document}
